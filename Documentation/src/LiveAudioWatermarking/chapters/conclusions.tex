\chapter{Conclusions}
To sum up the previous chapters and also give a brief recap it is possible to state that this project showed how watermarking and high frequency signals can be used to manipulate data.




This project, through the studying of the state of the art of the current technologies and techniques, took into account the weaknesses of such technologies and how concepts like non linearity could be used to obtain useful and interesting results regarding this study on security over voice communication.



High frequency signals, at first, were pretty useful at least in a theoretical way, with more
powerful equipment and a refined code it is possible to eventually inject data into these
communications and this can have a double effect: a malicious attacker can modify data and alter
the whole meaning of a phrase. Otherwise a "signature" can be embedded in the signal to certificate the source and the content of the signal.



The concept of watermarking, thanks to its properties, came to hand and was put into practice in an early version to be the "signature" cited above. The idea was to use watermarking as an assurance of authenticity and consistency of the information and guarantee the integrity of the information transmitted through common modern day microphones.



To implement all these concepts it became necessary to develop an Android app, this app was designed to be simple and easy to use, also a Matlab script has been wrote to analyze the correctness of the watermarking applied to the signal and to visualize its frequency spectrum. 


The long term goal of this project can be to improve the equipment and use this work as a starting point for further and more refined implementations of this approach towards audio watermarking for authentication purposes.

